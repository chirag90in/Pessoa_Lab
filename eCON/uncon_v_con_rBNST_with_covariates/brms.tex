\PassOptionsToPackage{unicode=true}{hyperref} % options for packages loaded elsewhere
\PassOptionsToPackage{hyphens}{url}
%
\documentclass[
]{article}
\usepackage{lmodern}
\usepackage{amssymb,amsmath}
\usepackage{ifxetex,ifluatex}
\ifnum 0\ifxetex 1\fi\ifluatex 1\fi=0 % if pdftex
  \usepackage[T1]{fontenc}
  \usepackage[utf8]{inputenc}
  \usepackage{textcomp} % provides euro and other symbols
\else % if luatex or xelatex
  \usepackage{unicode-math}
  \defaultfontfeatures{Scale=MatchLowercase}
  \defaultfontfeatures[\rmfamily]{Ligatures=TeX,Scale=1}
\fi
% use upquote if available, for straight quotes in verbatim environments
\IfFileExists{upquote.sty}{\usepackage{upquote}}{}
\IfFileExists{microtype.sty}{% use microtype if available
  \usepackage[]{microtype}
  \UseMicrotypeSet[protrusion]{basicmath} % disable protrusion for tt fonts
}{}
\makeatletter
\@ifundefined{KOMAClassName}{% if non-KOMA class
  \IfFileExists{parskip.sty}{%
    \usepackage{parskip}
  }{% else
    \setlength{\parindent}{0pt}
    \setlength{\parskip}{6pt plus 2pt minus 1pt}}
}{% if KOMA class
  \KOMAoptions{parskip=half}}
\makeatother
\usepackage{xcolor}
\IfFileExists{xurl.sty}{\usepackage{xurl}}{} % add URL line breaks if available
\IfFileExists{bookmark.sty}{\usepackage{bookmark}}{\usepackage{hyperref}}
\hypersetup{
  pdftitle={R Notebook},
  pdfborder={0 0 0},
  breaklinks=true}
\urlstyle{same}  % don't use monospace font for urls
\usepackage[margin=1in]{geometry}
\usepackage{color}
\usepackage{fancyvrb}
\newcommand{\VerbBar}{|}
\newcommand{\VERB}{\Verb[commandchars=\\\{\}]}
\DefineVerbatimEnvironment{Highlighting}{Verbatim}{commandchars=\\\{\}}
% Add ',fontsize=\small' for more characters per line
\usepackage{framed}
\definecolor{shadecolor}{RGB}{248,248,248}
\newenvironment{Shaded}{\begin{snugshade}}{\end{snugshade}}
\newcommand{\AlertTok}[1]{\textcolor[rgb]{0.94,0.16,0.16}{#1}}
\newcommand{\AnnotationTok}[1]{\textcolor[rgb]{0.56,0.35,0.01}{\textbf{\textit{#1}}}}
\newcommand{\AttributeTok}[1]{\textcolor[rgb]{0.77,0.63,0.00}{#1}}
\newcommand{\BaseNTok}[1]{\textcolor[rgb]{0.00,0.00,0.81}{#1}}
\newcommand{\BuiltInTok}[1]{#1}
\newcommand{\CharTok}[1]{\textcolor[rgb]{0.31,0.60,0.02}{#1}}
\newcommand{\CommentTok}[1]{\textcolor[rgb]{0.56,0.35,0.01}{\textit{#1}}}
\newcommand{\CommentVarTok}[1]{\textcolor[rgb]{0.56,0.35,0.01}{\textbf{\textit{#1}}}}
\newcommand{\ConstantTok}[1]{\textcolor[rgb]{0.00,0.00,0.00}{#1}}
\newcommand{\ControlFlowTok}[1]{\textcolor[rgb]{0.13,0.29,0.53}{\textbf{#1}}}
\newcommand{\DataTypeTok}[1]{\textcolor[rgb]{0.13,0.29,0.53}{#1}}
\newcommand{\DecValTok}[1]{\textcolor[rgb]{0.00,0.00,0.81}{#1}}
\newcommand{\DocumentationTok}[1]{\textcolor[rgb]{0.56,0.35,0.01}{\textbf{\textit{#1}}}}
\newcommand{\ErrorTok}[1]{\textcolor[rgb]{0.64,0.00,0.00}{\textbf{#1}}}
\newcommand{\ExtensionTok}[1]{#1}
\newcommand{\FloatTok}[1]{\textcolor[rgb]{0.00,0.00,0.81}{#1}}
\newcommand{\FunctionTok}[1]{\textcolor[rgb]{0.00,0.00,0.00}{#1}}
\newcommand{\ImportTok}[1]{#1}
\newcommand{\InformationTok}[1]{\textcolor[rgb]{0.56,0.35,0.01}{\textbf{\textit{#1}}}}
\newcommand{\KeywordTok}[1]{\textcolor[rgb]{0.13,0.29,0.53}{\textbf{#1}}}
\newcommand{\NormalTok}[1]{#1}
\newcommand{\OperatorTok}[1]{\textcolor[rgb]{0.81,0.36,0.00}{\textbf{#1}}}
\newcommand{\OtherTok}[1]{\textcolor[rgb]{0.56,0.35,0.01}{#1}}
\newcommand{\PreprocessorTok}[1]{\textcolor[rgb]{0.56,0.35,0.01}{\textit{#1}}}
\newcommand{\RegionMarkerTok}[1]{#1}
\newcommand{\SpecialCharTok}[1]{\textcolor[rgb]{0.00,0.00,0.00}{#1}}
\newcommand{\SpecialStringTok}[1]{\textcolor[rgb]{0.31,0.60,0.02}{#1}}
\newcommand{\StringTok}[1]{\textcolor[rgb]{0.31,0.60,0.02}{#1}}
\newcommand{\VariableTok}[1]{\textcolor[rgb]{0.00,0.00,0.00}{#1}}
\newcommand{\VerbatimStringTok}[1]{\textcolor[rgb]{0.31,0.60,0.02}{#1}}
\newcommand{\WarningTok}[1]{\textcolor[rgb]{0.56,0.35,0.01}{\textbf{\textit{#1}}}}
\usepackage{graphicx,grffile}
\makeatletter
\def\maxwidth{\ifdim\Gin@nat@width>\linewidth\linewidth\else\Gin@nat@width\fi}
\def\maxheight{\ifdim\Gin@nat@height>\textheight\textheight\else\Gin@nat@height\fi}
\makeatother
% Scale images if necessary, so that they will not overflow the page
% margins by default, and it is still possible to overwrite the defaults
% using explicit options in \includegraphics[width, height, ...]{}
\setkeys{Gin}{width=\maxwidth,height=\maxheight,keepaspectratio}
\setlength{\emergencystretch}{3em}  % prevent overfull lines
\providecommand{\tightlist}{%
  \setlength{\itemsep}{0pt}\setlength{\parskip}{0pt}}
\setcounter{secnumdepth}{-2}
% Redefines (sub)paragraphs to behave more like sections
\ifx\paragraph\undefined\else
  \let\oldparagraph\paragraph
  \renewcommand{\paragraph}[1]{\oldparagraph{#1}\mbox{}}
\fi
\ifx\subparagraph\undefined\else
  \let\oldsubparagraph\subparagraph
  \renewcommand{\subparagraph}[1]{\oldsubparagraph{#1}\mbox{}}
\fi

% set default figure placement to htbp
\makeatletter
\def\fps@figure{htbp}
\makeatother


\title{R Notebook}
\author{}
\date{\vspace{-2.5em}}

\begin{document}
\maketitle

This is an \href{http://rmarkdown.rstudio.com}{R Markdown} Notebook.
When you execute code within the notebook, the results appear beneath
the code.

Try executing this chunk by clicking the \emph{Run} button within the
chunk or by placing your cursor inside it and pressing
\emph{Ctrl+Shift+Enter}.

\hypertarget{population-effect-for-right-bst}{%
\section{Population effect for right
BST}\label{population-effect-for-right-bst}}

\hypertarget{effect-contrast-of-shock-responses-between-uncontrol-and-control-paricipants}{%
\subsubsection{effect: contrast of shock responses between uncontrol and
control
paricipants}\label{effect-contrast-of-shock-responses-between-uncontrol-and-control-paricipants}}

\begin{Shaded}
\begin{Highlighting}[]
\KeywordTok{library}\NormalTok{(brms) }\CommentTok{# for the analysis}
\end{Highlighting}
\end{Shaded}

\begin{verbatim}
## Loading required package: Rcpp
\end{verbatim}

\begin{verbatim}
## Loading 'brms' package (version 2.12.0). Useful instructions
## can be found by typing help('brms'). A more detailed introduction
## to the package is available through vignette('brms_overview').
\end{verbatim}

\begin{verbatim}
## 
## Attaching package: 'brms'
\end{verbatim}

\begin{verbatim}
## The following object is masked from 'package:stats':
## 
##     ar
\end{verbatim}

\begin{Shaded}
\begin{Highlighting}[]
\KeywordTok{library}\NormalTok{(haven) }\CommentTok{# to load the SPSS .sav file}
\KeywordTok{library}\NormalTok{(tidyverse) }\CommentTok{# needed for data manipulation.}
\end{Highlighting}
\end{Shaded}

\begin{verbatim}
## -- Attaching packages ---------------------------------------------------------------------------------------------------------------------------------------- tidyverse 1.3.0 --
\end{verbatim}

\begin{verbatim}
## v ggplot2 3.3.0     v purrr   0.3.3
## v tibble  2.1.3     v dplyr   0.8.3
## v tidyr   1.0.2     v stringr 1.4.0
## v readr   1.3.1     v forcats 0.5.0
\end{verbatim}

\begin{verbatim}
## -- Conflicts ------------------------------------------------------------------------------------------------------------------------------------------- tidyverse_conflicts() --
## x dplyr::filter() masks stats::filter()
## x dplyr::lag()    masks stats::lag()
\end{verbatim}

\begin{Shaded}
\begin{Highlighting}[]
\KeywordTok{library}\NormalTok{(RColorBrewer) }\CommentTok{# needed for some extra colours in one of the graphs}
\KeywordTok{library}\NormalTok{(ggmcmc)}
\end{Highlighting}
\end{Shaded}

\begin{verbatim}
## Registered S3 method overwritten by 'GGally':
##   method from   
##   +.gg   ggplot2
\end{verbatim}

\begin{Shaded}
\begin{Highlighting}[]
\KeywordTok{library}\NormalTok{(ggthemes)}
\KeywordTok{library}\NormalTok{(ggridges)}
\end{Highlighting}
\end{Shaded}

\hypertarget{model}{%
\section{\texorpdfstring{\textbf{Model}}{Model}}\label{model}}

\[Y = N(\mu, \sigma_{\epsilon}^{2})\]
\[\mu = \alpha + \beta_{Tm}Tm + \beta_{TD}TD + \beta_{Sm}Sm+\beta_{Sd}Sd + \epsilon\]
Where Tm, Td, Sm, and Sd are covariates.

Tm: Trait mean Td: Triat difference (uncon - con) Sm: State mean Sd:
State difference (uncon - con) \(\alpha: mean effect\) \#
\textbf{Priors}

\[N(0,100): \alpha\] \[N(0,100): \beta_{Tm}\] \[N(0,100): \beta_{Td}\]
\[N(0,100): \beta_{Sm}\] \[N(0,100): \beta_{Sd}\]
\[Cauchy(0,100): \sigma_{\epsilon}\]

\begin{Shaded}
\begin{Highlighting}[]
\KeywordTok{setwd}\NormalTok{(}\StringTok{"C:/Users/Chirag/Box/Box/UMD/Project_UMD/eCON/RBA/uncon_v_con_rBNST_with_covariates"}\NormalTok{)}

\NormalTok{df <-}\StringTok{ }\KeywordTok{read.table}\NormalTok{(}\StringTok{'uncon_v_con_rBNST_with_covariates.txt'}\NormalTok{,}\DataTypeTok{header=}\OtherTok{TRUE}\NormalTok{)}
\NormalTok{prior1 <-}\StringTok{ }\KeywordTok{c}\NormalTok{(}\KeywordTok{prior}\NormalTok{(}\KeywordTok{normal}\NormalTok{(}\DecValTok{0}\NormalTok{,}\DecValTok{100}\NormalTok{),}\DataTypeTok{class=}\NormalTok{Intercept),}
            \KeywordTok{prior}\NormalTok{(}\KeywordTok{normal}\NormalTok{(}\DecValTok{0}\NormalTok{,}\DecValTok{100}\NormalTok{),}\DataTypeTok{class=}\NormalTok{b, }\DataTypeTok{coef=}\StringTok{"TRAITmean"}\NormalTok{),}
            \KeywordTok{prior}\NormalTok{(}\KeywordTok{normal}\NormalTok{(}\DecValTok{0}\NormalTok{,}\DecValTok{100}\NormalTok{),}\DataTypeTok{class=}\NormalTok{b, }\DataTypeTok{coef=}\StringTok{"TRAITdiff"}\NormalTok{),}
            \KeywordTok{prior}\NormalTok{(}\KeywordTok{normal}\NormalTok{(}\DecValTok{0}\NormalTok{,}\DecValTok{100}\NormalTok{),}\DataTypeTok{class=}\NormalTok{b, }\DataTypeTok{coef=}\StringTok{"STATEmean"}\NormalTok{),}
            \KeywordTok{prior}\NormalTok{(}\KeywordTok{normal}\NormalTok{(}\DecValTok{0}\NormalTok{,}\DecValTok{100}\NormalTok{),}\DataTypeTok{class=}\NormalTok{b, }\DataTypeTok{coef=}\StringTok{"STATEdiff"}\NormalTok{),}
            \KeywordTok{prior}\NormalTok{(}\KeywordTok{cauchy}\NormalTok{(}\DecValTok{0}\NormalTok{,}\DecValTok{100}\NormalTok{),}\DataTypeTok{class=}\NormalTok{sigma)}
\NormalTok{           )}

\NormalTok{bmod1 <-}\StringTok{ }\KeywordTok{brm}\NormalTok{(Y }\OperatorTok{~}\StringTok{ }\NormalTok{TRAITmean }\OperatorTok{+}\StringTok{ }\NormalTok{TRAITdiff }\OperatorTok{+}\StringTok{ }\NormalTok{STATEmean }\OperatorTok{+}\StringTok{ }\NormalTok{STATEdiff, }
             \DataTypeTok{data =}\NormalTok{ df, }
             \DataTypeTok{family =} \KeywordTok{gaussian}\NormalTok{(),}
             \DataTypeTok{prior =}\NormalTok{ prior1, }
             \DataTypeTok{warmup =} \DecValTok{2000}\NormalTok{, }\DataTypeTok{iter =} \DecValTok{5000}\NormalTok{,}
             \DataTypeTok{chains =} \DecValTok{4}\NormalTok{,}
             \DataTypeTok{cores  =} \DecValTok{2}\NormalTok{)}
\end{Highlighting}
\end{Shaded}

\begin{verbatim}
## Compiling the C++ model
\end{verbatim}

\begin{verbatim}
## Start sampling
\end{verbatim}

\begin{Shaded}
\begin{Highlighting}[]
\KeywordTok{summary}\NormalTok{(bmod1)}
\end{Highlighting}
\end{Shaded}

\begin{verbatim}
##  Family: gaussian 
##   Links: mu = identity; sigma = identity 
## Formula: Y ~ TRAITmean + TRAITdiff + STATEmean + STATEdiff 
##    Data: df (Number of observations: 61) 
## Samples: 4 chains, each with iter = 5000; warmup = 2000; thin = 1;
##          total post-warmup samples = 12000
## 
## Population-Level Effects: 
##           Estimate Est.Error l-95% CI u-95% CI Rhat Bulk_ESS Tail_ESS
## Intercept     0.11      0.06    -0.01     0.23 1.00    16944     9159
## TRAITmean    -0.05      0.08    -0.21     0.10 1.00    11048     8647
## TRAITdiff     0.07      0.07    -0.06     0.19 1.00    14548     9003
## STATEmean     0.21      0.08     0.06     0.36 1.00    11453     9415
## STATEdiff     0.05      0.06    -0.07     0.18 1.00    14508     8403
## 
## Family Specific Parameters: 
##       Estimate Est.Error l-95% CI u-95% CI Rhat Bulk_ESS Tail_ESS
## sigma     0.48      0.05     0.40     0.58 1.00    12983     8764
## 
## Samples were drawn using sampling(NUTS). For each parameter, Bulk_ESS
## and Tail_ESS are effective sample size measures, and Rhat is the potential
## scale reduction factor on split chains (at convergence, Rhat = 1).
\end{verbatim}

This is an \href{http://rmarkdown.rstudio.com}{R Markdown} Notebook.
When you execute code within the notebook, the results appear beneath
the code.

Try executing this chunk by clicking the \emph{Run} button within the
chunk or by placing your cursor inside it and pressing
\emph{Ctrl+Shift+Enter}.

\hypertarget{population-effect-for-right-bst-1}{%
\section{Population effect for right
BST}\label{population-effect-for-right-bst-1}}

\hypertarget{effect-contrast-of-shock-responses-between-uncontrol-and-control-paricipants-1}{%
\subsubsection{effect: contrast of shock responses between uncontrol and
control
paricipants}\label{effect-contrast-of-shock-responses-between-uncontrol-and-control-paricipants-1}}

\begin{Shaded}
\begin{Highlighting}[]
\KeywordTok{library}\NormalTok{(brms) }\CommentTok{# for the analysis}
\end{Highlighting}
\end{Shaded}

\begin{verbatim}
## Loading required package: Rcpp
\end{verbatim}

\begin{verbatim}
## Loading 'brms' package (version 2.12.0). Useful instructions
## can be found by typing help('brms'). A more detailed introduction
## to the package is available through vignette('brms_overview').
\end{verbatim}

\begin{verbatim}
## 
## Attaching package: 'brms'
\end{verbatim}

\begin{verbatim}
## The following object is masked from 'package:stats':
## 
##     ar
\end{verbatim}

\begin{Shaded}
\begin{Highlighting}[]
\KeywordTok{library}\NormalTok{(haven) }\CommentTok{# to load the SPSS .sav file}
\KeywordTok{library}\NormalTok{(tidyverse) }\CommentTok{# needed for data manipulation.}
\end{Highlighting}
\end{Shaded}

\begin{verbatim}
## -- Attaching packages ---------------------------------------------------------------------------------------------------------------------------------------- tidyverse 1.3.0 --
\end{verbatim}

\begin{verbatim}
## v ggplot2 3.3.0     v purrr   0.3.3
## v tibble  2.1.3     v dplyr   0.8.3
## v tidyr   1.0.2     v stringr 1.4.0
## v readr   1.3.1     v forcats 0.5.0
\end{verbatim}

\begin{verbatim}
## -- Conflicts ------------------------------------------------------------------------------------------------------------------------------------------- tidyverse_conflicts() --
## x dplyr::filter() masks stats::filter()
## x dplyr::lag()    masks stats::lag()
\end{verbatim}

\begin{Shaded}
\begin{Highlighting}[]
\KeywordTok{library}\NormalTok{(RColorBrewer) }\CommentTok{# needed for some extra colours in one of the graphs}
\KeywordTok{library}\NormalTok{(ggmcmc)}
\end{Highlighting}
\end{Shaded}

\begin{verbatim}
## Registered S3 method overwritten by 'GGally':
##   method from   
##   +.gg   ggplot2
\end{verbatim}

\begin{Shaded}
\begin{Highlighting}[]
\KeywordTok{library}\NormalTok{(ggthemes)}
\KeywordTok{library}\NormalTok{(ggridges)}
\end{Highlighting}
\end{Shaded}

\hypertarget{model-1}{%
\section{\texorpdfstring{\textbf{Model}}{Model}}\label{model-1}}

\[Y = N(\mu, \sigma_{\epsilon}^{2})\]
\[\mu = \alpha + \beta_{Tm}Tm + \beta_{TD}TD + \beta_{Sm}Sm+\beta_{Sd}Sd + \epsilon\]
Where Tm, Td, Sm, and Sd are covariates.

Tm: Trait mean Td: Triat difference (uncon - con) Sm: State mean Sd:
State difference (uncon - con) \(\alpha: mean effect\) \#
\textbf{Priors}

\[N(0,100): \alpha\] \[N(0,100): \beta_{Tm}\] \[N(0,100): \beta_{Td}\]
\[N(0,100): \beta_{Sm}\] \[N(0,100): \beta_{Sd}\]
\[Cauchy(0,100): \sigma_{\epsilon}\]

\begin{Shaded}
\begin{Highlighting}[]
\KeywordTok{setwd}\NormalTok{(}\StringTok{"C:/Users/Chirag/Box/Box/UMD/Project_UMD/eCON/RBA/uncon_v_con_rBNST_with_covariates"}\NormalTok{)}

\NormalTok{df <-}\StringTok{ }\KeywordTok{read.table}\NormalTok{(}\StringTok{'uncon_v_con_rBNST_with_covariates.txt'}\NormalTok{,}\DataTypeTok{header=}\OtherTok{TRUE}\NormalTok{)}
\NormalTok{prior1 <-}\StringTok{ }\KeywordTok{c}\NormalTok{(}\KeywordTok{prior}\NormalTok{(}\KeywordTok{normal}\NormalTok{(}\DecValTok{0}\NormalTok{,}\DecValTok{100}\NormalTok{),}\DataTypeTok{class=}\NormalTok{Intercept),}
            \KeywordTok{prior}\NormalTok{(}\KeywordTok{normal}\NormalTok{(}\DecValTok{0}\NormalTok{,}\DecValTok{100}\NormalTok{),}\DataTypeTok{class=}\NormalTok{b, }\DataTypeTok{coef=}\StringTok{"TRAITmean"}\NormalTok{),}
            \KeywordTok{prior}\NormalTok{(}\KeywordTok{normal}\NormalTok{(}\DecValTok{0}\NormalTok{,}\DecValTok{100}\NormalTok{),}\DataTypeTok{class=}\NormalTok{b, }\DataTypeTok{coef=}\StringTok{"TRAITdiff"}\NormalTok{),}
            \KeywordTok{prior}\NormalTok{(}\KeywordTok{normal}\NormalTok{(}\DecValTok{0}\NormalTok{,}\DecValTok{100}\NormalTok{),}\DataTypeTok{class=}\NormalTok{b, }\DataTypeTok{coef=}\StringTok{"STATEmean"}\NormalTok{),}
            \KeywordTok{prior}\NormalTok{(}\KeywordTok{normal}\NormalTok{(}\DecValTok{0}\NormalTok{,}\DecValTok{100}\NormalTok{),}\DataTypeTok{class=}\NormalTok{b, }\DataTypeTok{coef=}\StringTok{"STATEdiff"}\NormalTok{),}
            \KeywordTok{prior}\NormalTok{(}\KeywordTok{cauchy}\NormalTok{(}\DecValTok{0}\NormalTok{,}\DecValTok{100}\NormalTok{),}\DataTypeTok{class=}\NormalTok{sigma)}
\NormalTok{           )}

\NormalTok{bmod1 <-}\StringTok{ }\KeywordTok{brm}\NormalTok{(Y }\OperatorTok{~}\StringTok{ }\NormalTok{TRAITmean }\OperatorTok{+}\StringTok{ }\NormalTok{TRAITdiff }\OperatorTok{+}\StringTok{ }\NormalTok{STATEmean }\OperatorTok{+}\StringTok{ }\NormalTok{STATEdiff, }
             \DataTypeTok{data =}\NormalTok{ df, }
             \DataTypeTok{family =} \KeywordTok{gaussian}\NormalTok{(),}
             \DataTypeTok{prior =}\NormalTok{ prior1, }
             \DataTypeTok{warmup =} \DecValTok{2000}\NormalTok{, }\DataTypeTok{iter =} \DecValTok{5000}\NormalTok{,}
             \DataTypeTok{chains =} \DecValTok{4}\NormalTok{,}
             \DataTypeTok{cores  =} \DecValTok{2}\NormalTok{)}
\end{Highlighting}
\end{Shaded}

\begin{verbatim}
## Compiling the C++ model
\end{verbatim}

\begin{verbatim}
## Start sampling
\end{verbatim}

\begin{Shaded}
\begin{Highlighting}[]
\KeywordTok{summary}\NormalTok{(bmod1)}
\end{Highlighting}
\end{Shaded}

\begin{verbatim}
##  Family: gaussian 
##   Links: mu = identity; sigma = identity 
## Formula: Y ~ TRAITmean + TRAITdiff + STATEmean + STATEdiff 
##    Data: df (Number of observations: 61) 
## Samples: 4 chains, each with iter = 5000; warmup = 2000; thin = 1;
##          total post-warmup samples = 12000
## 
## Population-Level Effects: 
##           Estimate Est.Error l-95% CI u-95% CI Rhat Bulk_ESS Tail_ESS
## Intercept     0.11      0.06    -0.01     0.23 1.00    16944     9159
## TRAITmean    -0.05      0.08    -0.21     0.10 1.00    11048     8647
## TRAITdiff     0.07      0.07    -0.06     0.19 1.00    14548     9003
## STATEmean     0.21      0.08     0.06     0.36 1.00    11453     9415
## STATEdiff     0.05      0.06    -0.07     0.18 1.00    14508     8403
## 
## Family Specific Parameters: 
##       Estimate Est.Error l-95% CI u-95% CI Rhat Bulk_ESS Tail_ESS
## sigma     0.48      0.05     0.40     0.58 1.00    12983     8764
## 
## Samples were drawn using sampling(NUTS). For each parameter, Bulk_ESS
## and Tail_ESS are effective sample size measures, and Rhat is the potential
## scale reduction factor on split chains (at convergence, Rhat = 1).
\end{verbatim}

\end{document}
